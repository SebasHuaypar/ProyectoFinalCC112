\documentclass[twocolumn]{article}
\usepackage{graphicx}
\usepackage{lipsum} % Paquete para generar texto de prueba
\usepackage{listings} % Paquete para incluir código

\begin{document}

\begin{titlepage}
    \centering
    {\Huge \bfseries Proyecto Final CC112 \par}
    \vspace{2cm}
    \includegraphics[width=0.5\textwidth]{uni_logo.png} 
    \vfill
    {\Large Huaypar Acurio Sebastián Enrique \par}
    \vspace{0.5cm}
    {\large Escuela de Ciencia de la Computación \par}
    \vspace{0.5cm}
    {\large sebastian.huaypar.a@uni.pe \par}
    \vspace{0.5cm}
    {\large Lima, Perú \par}
\end{titlepage}

\onecolumn % Cambiamos a una sola columna para la página de resumen
\begin{abstract}
El presente informe detalla el desarrollo del proyecto final del curso CC112, centrado en el uso de Python para resolver problemas y automatizar procesos. Se explora la implementación de ejercicios básicos, una calculadora de tipo de cambio y la automatización de tareas web utilizando herramientas como Tkinter, Ipywidgets y Selenium.
\end{abstract}

\twocolumn % Volvemos a dos columnas para el resto del documento

\section*{Introducción}
Python se ha destacado como un lenguaje versátil y poderoso, utilizado ampliamente en diversas áreas gracias a su sintaxis clara, extensa comunidad de desarrolladores y abundante disponibilidad de bibliotecas. En este informe se detalla el desarrollo de un proyecto utilizando Python, enfocado en la programación básica, aplicaciones con interfaces gráficas y automatización de tareas web utilizando herramientas específicas.

\section*{Marco Teórico}
Para el desarrollo de este proyecto, se utilizaron diferentes herramientas y entornos de desarrollo. Se empleó Visual Studio Code (VS Code) como el principal entorno de desarrollo integrado (IDE), conocido por su robustez y extensibilidad mediante plugins. Además, se utilizó Google Colab para ejecutar y colaborar en scripts de Python de manera colaborativa en la nube.

Dentro de los scripts desarrollados, se implementaron las siguientes bibliotecas y herramientas:
\begin{itemize}
    \item \textbf{Tkinter}: Utilizada para la creación de interfaces gráficas de usuario (GUI) en Python.
    \item \textbf{Ipywidgets}: Empleada en entornos Jupyter Notebooks para crear widgets interactivos.
    \item \textbf{Selenium}: Utilizada para la automatización de navegadores web.
    \item \textbf{Webdriver\_manager}: Herramienta para la gestión automática de controladores (drivers) de navegadores.
\end{itemize}

\section*{Desarrollo}

\subsection*{Parte 1: Ejercicios Básicos en Python}
En la primera parte del proyecto se desarrollaron cinco ejercicios fundamentales en Python:

\subsubsection*{Ejercicio 1: Cálculo del Factorial}
Este ejercicio consistió en crear una función que calcule el factorial de un número utilizando bucles.

\subsubsection*{Ejercicio 3: Inversión de Cadena}
Se implementó una función que invierte una cadena de texto.

\subsubsection*{Ejercicio 5: Mutabilidad de Listas en Python}
Se exploraron las referencias y mutabilidad en Python, demostrando cómo las listas pueden ser modificadas dentro de funciones.

\subsubsection*{Ejercicio 7: Agenda Telefónica con Diccionarios}
Se desarrolló una agenda telefónica utilizando diccionarios para almacenar nombres y números de teléfono.

\subsubsection*{Ejercicio 9: Lectura y Escritura de Archivos de Texto}
Se escribieron datos en un archivo de texto y se leyeron posteriormente utilizando funciones de manejo de archivos en Python.

\subsection*{Parte 2: Implementación de una Calculadora de Tipo de Cambio}
Se implementó un programa utilizando dos enfoques:

\subsubsection*{Interfaz Gráfica con Tkinter}
Se creó una interfaz gráfica utilizando Tkinter para un programa descargable que convierte entre diferentes monedas utilizando tasas de cambio predefinidas.

\subsubsection*{Ejecución en Google Colab con Ipywidgets}
Se adaptó el mismo programa para ejecutarse en Google Colab utilizando Ipywidgets, permitiendo la interacción y ejecución en un entorno colaborativo en la nube.

\subsection*{Parte 3: Automatización con Selenium}
Se desarrolló un script utilizando Selenium para automatizar tareas en un navegador web. El script fue capaz de llenar y enviar automáticamente un formulario en línea, demostrando el uso de Selenium para la automatización de procesos web interactivos.

\section*{Conclusiones}
El desarrollo de este proyecto proporcionó una sólida comprensión de los conceptos fundamentales de Python, así como la capacidad de integrar bibliotecas y herramientas para resolver problemas reales. La combinación de ejercicios prácticos, implementaciones de proyectos y automatización de procesos web no solo fortaleció las habilidades de programación, sino que también amplió el conocimiento sobre el uso de Python en diferentes contextos.

El uso de herramientas como Tkinter, Ipywidgets y Selenium permitió explorar las capacidades de Python en la creación de interfaces gráficas, el manejo de datos externos y la automatización de tareas repetitivas, demostrando la versatilidad y eficiencia del lenguaje en la resolución de problemas complejos.

\end{document}

